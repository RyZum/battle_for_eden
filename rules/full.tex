\documentclass[a4paper]{article}
\usepackage{fullpage}


\begin{document}

\title{Battle For Eden Rulebook}
\maketitle


    Civilizations rise and fall across the multiverse.

    Meanwhile, a place is left untouched, never knowing magic or technology.
    Its peaceful inhabitants live in harmony with nature, never knowing hunger or lacking anything, for it is a land of abundance.
    A land filled with all kinds of riches, a land filled with all kinds of resources,
    and above all a land filled with rare magic crystals: a land called Eden!

    When the mana crystals become so numerous that portals to Eden suddendly open throughout the multiverse,
    it is time for civilizations to go to war! War for crystals! War for resources!
    This is the battle for Eden!



\section{Introduction}

    Battle For Eden is a war game involving conquer, resource production and deck building.
    Each player embodies a different civilization, and as such uses different cards and relies on
    different resources than other players, giving birth to an asymetric game with a lot of replayability!

    As civilizations discover Eden, they may face various situations.
    This game comes with various scenarios, each with its own victory conditions.
    Some scenarios will pitch players against one another, while other will have them cooperate to survive!
    Whichever scenario you choose, fight for the prosperity of your civilization!



\section{Contents}

    \begin{itemize}
        \item This rulebook
        \item Civilization unit cards
            \begin{itemize}
                \item 8 Basic Unit cards
                \item 6 Shock Unit cards
                \item 4 Warrior Unit cards
                \item 4 Tactical Unit cards
                \item 3 Champion Unit cards
                \item 2 Immortal Unit cards
            \end{itemize}
        \item Common cards
            \begin{itemize}
                \item Mana Crystal cards
                    \begin{itemize}
                        \item 30 Small Mana Crystal cards
                        \item 20 Medium Mana Crystal cards
                        \item 20 Huge Mana Crystal cards
                    \end{itemize}
                \item Tactical Advantage cards
                    \begin{itemize}
                        \item 30 Small Mana Crystal cards
                        \item 20 Medium Mana Crystal cards
                        \item 20 Huge Mana Crystal cards
                        \item 20 Setback cards
                        \item 20 Opportunity cards
                    \end{itemize}
            \end{itemize}
\newpage
        \item Territoriy tiles
            \begin{itemize}
                \item 1 Heart of Eden tile
                \item 6 Portal tiles (1 for each civilization)
                \item 74 normal territory tiles
            \end{itemize}
        \item Occupation tokens
        \item Resource tokens
        \item Material for the King of Eden scenario
            \begin{itemize}
                \item 1 King of Eden figurine
                \item 1 20-faced dice
                \item an attack deck
            \end{itemize}
    \end{itemize}



\section{Setup}

    \begin{description}
        \item[Civilization choice] \hfill \\
            Each player chooses a civilization.
            He takes all unit cards of that civilization and puts them in front of him as piles of each type of unit.
            Those cards are called his World.
            The player also chooses a color and takes the corresponding Occupation tokens.
        \item[Map creation] \hfill \\
            Place the Heart of Eden tile at the center of the table.
            Shuffle normal territory tiles.
            For each player, place his civilization's portal tile on the table, and use territory tiles
            to connect it to the Heart of Eden in such a way that there are 3 territories between any Portal and the Heart of Eden.
            Add territory tiles so that each Portal is in contact with 2 territories, producing exactly 1 of the valuable
            resource of the corresponding civilization (see section on resource production).
            Add territory tiles to the map at you convenance. Some map examples are available at the end of this book.
            Each player places an Occupation token on his portal and the 2 territories in contact with it,
            then a valuable resource token on those 2 territories.
        \item[Common cards] \hfill \\
            Create a pile of each type of common card, and place them all in the central area, so that
            every player can see them.
        \item[Deck creation] \hfill \\
            Each player takes 5 Small Mana Crystal cards from the central area and
            3 Basic Units cards from his World, then shuffle them to create his deck.
        \item[Hand creation] \hfill \\
            Each player draws 5 cards from his deck to create his hand.
            If his hand only contains Small Mana Crystals, he shows it to the other players, shuffles it back into his deck
            and does the Hand creation step once more.
    \end{description}



\section{A player's turn}

    During his turn, a player can perform several actions, which can be done in any order but not simultaneously:

    \begin{itemize}
        \item Discard any card from his hand
        \item Produce resources on his territories (only once per turn)
        \item Buy cards
        \item Launching attacks on territories (only once per turn)
        \item Following through attacks (only after successfully conquering territories)
    \end{itemize}

    At the end of a player's turn, any player whose hand contain less than 5 cards can replenish his hand to 5 cards.


\newpage
\section{Drawing and discarding cards}

    When a card is discarded, it goes on top of its owner's discard pile, a pile of face-up cards beside his deck.

    When a player must draw more cards from his deck (whether to replenish his hand or because of a card's effect)
    or reveal more cards from the top of his deck than his deck contains cards, that player must replenish his deck before
    drawing or revealing the cards.
    To do so, he shuffles his discard pile and puts the cards under his deck.



\section{Producing resources}

    A player's Occupation tokens are used to materialize which territories he controls.
    Once per turn, a player can have all his territories produce resources.

    Except Portal territores, all territories can produce resources, which can be of 7 types:

    \begin{itemize}
        \item Mana Shards
        \item Plants
        \item Animals
        \item Ore
        \item Gems
        \item Rare Elements
        \item Inhabitant
    \end{itemize}

    The resource represented on a civilization's Portal tile is called the "valuable resource" of this civilization and is a necessary for its development (see "Buying cards").

    On each terrotory tile (except Portal tiles), the number of resources of each type it can produce is written.
    When producing resources, a player chooses a type of resource to produce for each of his territories and places a corresponding number of resource tokens of that type on the territory.

    IMPORTANT: territories outside of the map are considered to be able to produce 1 resource token of each kind.



\section{Buying cards}

    Buying cards is the action of paying a cost to get a card from your world or the central area.

    If the card does not say otherwise, the bought card is put in the discard pile of the player who bought it.

    When buying cards from your World ("asking for reinforcements"),
    the cost must be payed by discarding Mana Crystal cards of value equal to or greater
    than the total cost of the bought cards.

    When buying Mana Crystal cards or Tactical Advantage cards,
    the cost must be payed by removing that number of Mana Shard tokens from your territories.

    In all cases, buying a card also requires to pay an additional cost by removing as
    many valuable resource tokens from your territories as the cost of the card.

    For example, "buying a Medium Mana Crystal card" means removing 5 Mana Shard tokens and 5 valuable ressource
    tokens from your territories to take a Medium Mana Crystal card from the central area and put it in your discard pile.

    IMPORTANT: A player can only ask for reinforcements if he controls his Portal territory. 


\newpage
\section{Launching attacks on territories and following through}

    Once per turn, a player may choose frontiers between territories he controls and other territories and
    start a battle on each of those frontiers.
    Each of those battles takes place on a separate "front" (see next section for more information on combat).

    Once all battles are over, if the attacking player won battles on some fronts, he takes control of
    the corresponding territories by removing the Occupation tokens of the previous controller of those territories (if any)
    and putting his own Occupation tokens on the conquered territories.
    Resource tokens stay on conquered territories.
    
    IMPORTANT: this means that winning on one front is enough to take control of a territory.

    If the conquered territory was a Portal territory, the tile is instead removed from the map.
    A player who lost his Portal can keep playing until he controls no territory.

    If at least one territory was conquered, the attacking player earns the right to follow through his attack.
    Other actions may be performed before following through the attack.

    Following through one's attacks is basically the same thing as launching attacks, with only two differences:

    \begin{itemize}
        \item Only frontiers with the newly conquered territories can be chosen.
        \item If there is at least 1 defending player, all defending players draw 2 cards, and the attacking player draws 1 card.
    \end{itemize}

    If new territories are conquered this way, the attacking player earns the right to another "Follow through" action.



\section{Combat}

    When launching attacks, the attacking player adds a unit from his hand to all fronts.
    Then, on each front, defending players (if any) either stop defending or add a unit from their hands.
    If a defending player stops defending a front, he can't add any more units on this front.
    
    The attacking and defending players take turn in adding units to all fronts or stop adding units on those fronts.
    When all players have stopped adding units on all fronts, each player computes the total strength of his army on
    each front and compares it to the opposing army's strength.
    If one side's army is stronger than the other, the battle on this front ends as the victory of this side.
    Otherwise, the battle on this front ends as a draw.
    
    After the battles comes the aftermath.
    Units that have been doomed are destroyed (see next section on abilities).
    Other units are discarded.
    
    IMPORTANT: if there is no defending player (i.e. the attacked territory is a wild territory),
        the attacking player is the only one adding units on the front.
        He then compares his army's strength to the natural defenses of the territory to determine if he won or not.
        A draw against a wild territory is considered a victory.
        The natural defenses of a wild territory are equal to the distance of this territory to the closest Portal
        (1 if the territory's tile is in contact with a Portal tile, 2 if it's in contact with a tile which is in contact with a Portal tile, etc).


\newpage
\section{Abilities}

    Units played during combat don't simply add strength to their side.
    They also have abilities that greatly impact how the battle unfolds.
    Those abilities can be permanent abilities or triggered abilities.

    \begin{itemize}
        \item Permanent abilities are additional rules to the battle that take effect as long
            as units possessing them this ability is participating in combat.
            For example, "This unit cannt be destroyed" and "+2 strength when wounded" are permanent abilities.
        \item Triggered abilities are effects that can be activated once by the player at specific times.
            The player can always choose not to trigger an ability.
            There are 4 types of triggered abilities:
            \begin{itemize}
        	        \item Immediate abilities may be triggered just after the unit is added to the battle.
        	        \item Delayed abilities may be triggered when all players stopped attacking/defending,
	            before the actual end of the battle. The attacking units' delayed abilities are triggered first.
        	        \item Reaction abilities may be event-based or condition-based.
	            \begin{itemize}
                        \item Event-based reaction abilities may be triggered each time a specific event occurs
                            (for example "when a unit is destroyed").
                        \item Condition-based reaction abilities may be triggered just after the unit has been played
                            if a specific condition is satisfied (for example: "no cards in hand"), and may also be triggered
                            each time the condition becomes satisfied when it was not.
                    \end{itemize}
        	        \item Aftermath abilities may be triggered after the battles, during the aftermath.
            \end{itemize}
    \end{itemize}

    Some abilities allow a player to "destroy" a card.
    This means putting the card back on the pile it was bought from.
    Other abilities allow a player to "doom" a unit.
    A doomed unit is turned horizontally so that players don't forget it is doomed.
    During the aftermath, all doomed units are destroyed.
    
    IMPORTANT: Destroyed Immortal Unit cards are discarded instead of put back on the pile they were bought from.
    
    IMPORTANT: Some abilities allow a player to destroy a card from his hand, his deck, his discard pile, or one of his units in battle.
        Those destruction can only take place against an opponent.





\end{document}


\documentclass[a4paper]{article}

\usepackage[english]{babel}
\usepackage[utf8x]{inputenc}
\usepackage{indentfirst}

\author{Adrien Matricon}

\setlength{\parskip}{1em}

\begin{document}

\title{Battle For Eden Rulebook}
\maketitle


    Civilizations rise and fall across the multiverse.

    Meanwhile, a place is left untouched, never knowing magic or technology.
    Its peaceful fairy inhabitants live in harmony with nature,
    never knowing hunger or lacking anything, for it is a land of abundance.
    A land filled with all kinds of riches, a land filled with all kinds of resources,
    and above all a land filled with rare energy crystals: a land called Eden!

    When energy crystals become so numerous that portals to Eden suddendly open
    throughout the multiverse, it is time for civilizations to go to war!
    War for crystals! War for resources! This is the battle for Eden!



\section{Introduction}

    Battle For Eden is a war game involving conquest, resource production and deck
    building.
    Each player embodies a different civilization, and as such uses different cards and
    relies on different resources than other players, giving birth to an asymetric game
    with a lot of replayability!

    As civilizations discover Eden, they may face various situations.
    This game comes with various scenarios, each with its own victory conditions.
    Some scenarios will pitch players against one another, while other will have
    them cooperate to survive!
    Whichever scenario you choose, fight for the prosperity of your civilization!


\newpage
\section{Contents}

    \begin{itemize}
        \item This rulebook
        \item Civilization unit cards
            \begin{itemize}
                \item 8 cost-1 unit cards
                \item 6 cost-2 unit cards
                \item 4x4 cost-3 unit cards
                \item 3 cost-5 unit cards
                \item 2 cost-6 unit cards (eternal unit cards)
            \end{itemize}
        \item Common cards
            \begin{itemize}
                \item Energy Crystal cards
                    \begin{itemize}
                        \item 30 Small Energy Crystal cards
                        \item 20 Medium Energy Crystal cards
                        \item 20 Huge Energy Crystal cards
                    \end{itemize}
                \item Tactical cards
                    \begin{itemize}
                        \item 20 Setback cards
                        \item 20 Opportunity cards
                    \end{itemize}
            \end{itemize}
        \item Territoriy tiles
            \begin{itemize}
                \item 1 Heart of Eden tile
                \item 6 Portal tiles (1 for each civilization)
                \item 74 normal tiles
            \end{itemize}
        \item Occupation tokens
        \item Resource tokens
        \item Counters
        \item Material for the King of Eden scenario
            \begin{itemize}
                \item 1 King of Eden figurine
                \item 1 20-faced dice
                \item an attack deck
            \end{itemize}
    \end{itemize}


\newpage
\section{Setup}

    \begin{description}
        \item[Civilization choice] \hfill \\
            Each player chooses a civilization, and takes the corresponing
            \textbf{portal tile}, \textbf{occupation tokens} and \textbf{unit cards}.
            He then those cards in front of him as a pile, face up.
            This pile is called his \textit{\textbf{world}}.
        \item[Deck creation] \hfill \\
            Each player takes \textbf{3 Small Energy Crystal cards}
            and \textbf{1 Medium Energy Crystal cards}
            from the corresponding pile and \textbf{3 cost-1 units cards} from his World,
            then \textbf{shuffle them} to create his deck.
        \item[Choosing a first player] \hfill \\
            Players decide who the first player should be.
            
            By default, the last player to have lost a game
            (of any kind) is the first player.
    \end{description}

\newpage
\section{Some definitions}

    \begin{description}
        \item[Round] \hfill \\
            Each phase of the game is composed of rounds,
            during which each player plays one turn, one after the other,
            always in the same order.
        \item[Distance between two tiles] \hfill \\
            The number of frontiers one needs to cross to go from one tile to the another.
            For example, the distance between two tiles in contact is 1.
        \item[Discard pile] \hfill \\
            A pile of face-up cards (empty at the begging) besides a player's deck.
            Any player may look into any other player's discard pile.
            No player is allowed to change the order of the cards in any discard pile.
        \item[Discarding a card] \hfill \\
            Putting a card on top of its owner's discard pile.
        \item[Territory controlled by a player] \hfill \\
            Territory on which that player has put an Occupation token.
        \item[Territory fully controlled by a player] \hfill \\
            Territory controlled by a player and such that there exists a path
            between that territory and the player's Portal territory that
            goes only through that player's controlled territories.
        \item[Territory partially controlled by a player] \hfill \\
            Territory controlled by a player but not fully controlled by that player.
        \item[Wild territory] \hfill \\
            Territory controlled by no player.
        \item[Using resources] \hfill \\
            Removing a given number of resource tokens from one's fully controlled
            territories.
        \item[Using Energy Crystals] \hfill \\
            Discarding Energy Crystal cards whose total value is at least a given number.
        \item[Buying a card] \hfill \\
            Paying a cost to add a card to one's discard pile.
        \item[Destroying a card] \hfill \\
            Putting a card in the pile where it can be bought back.
            Destroyed common cards go into the corresponding piles,
            while destroyed unit cards go back to their World.
    \end{description}


\section{Map creation phase}

    The first phase of the game is the creation of the map.
    
    Each player takes one tile of each color from the game box,
    then adds to those a tile of the color of his valuable resource
    (todo: introduce term earlier?).
    He shuffles all those tiles together,
    and places them on the table face down as a pile.
    
    The Heart of Eden tile is placed at the center of the table.
    Then, starting with the first player, all players take turns building the map.
    Each turn starts by revealing the top tile of one's pile then placing it on the map.
    The first tile must be placed next to the Heart of Eden tile,
    then each tile may be placed anywhere next to another normal tile,
    as long as it leaves the map in a \textbf{valid state} (see below).
    In cooperative game modes, players decide where to put tiles before revealing them.
    
    If a tile comes to be in contact with 4 or more types of normal tiles,
    place an increased-shard-production token on that tile.
    
    At any time during his turn, a player is allowed to remove his Portal tile
    from the map (if he already placed it) and to place it next to two
    adjacent normal tiles, as long as it leaves the map in a
    \textbf{valid state} state (see below).  
    
    A \textbf{\textit{valid state}} of the map is one in which:
    \vspace{-1.3em}
    \begin{itemize}
        \item Every Portal tile is in contact with exactly 2 tiles,
        \item The distance between any 2 Portal tiles is 4 or more,
        \item The distance between any Portal tile and the Heart of Eden tile
        is 4 or more.
    \end{itemize}
    
    \vspace{-0.7em}
    During the last round, after playing his last normal tile,
    any player who has not placed his Portal tile must place it on the map
    if it is possible.
    Once all tile piles are empty, if there still are players who cannot
    place their Portal tiles, all tile piles are refilled with an additional tile
    and an additional round is played, during which normal tiles
    are now placed \textbf{face down}.
    Each additional round is played as if it was the last,
    and leads to an additional round if there are still players that cannot place
    their Portal tiles.
    


\newpage
\section{Main phase}

  \subsection{Beginning}
  
  	\begin{itemize}
        \item Create a pile of each type of common card,
            and place them somewhere where every player can access them.
        \item Players take control of the territories in contact with their Portal
        territory, and materialize it by putting Occupation tokens on them.
        \item Players may \textbf{manage their territories} (see below).
        \item Player draw 5 cards from their decks to constitute their starting hands.
    \end{itemize}
    
  \subsection{The pulse}
  
  	At the end of each round (\textit{i.e.} after each turn of the last player),
  	the Heart of Eden \textit{pulses}.
  	Each player adds the value of 1 in counter tokens to his Portal territory.
  	If a player controls the Heart of Eden, he add the value of 2
        to his Portal territory instead.
  	No token is placed on destroyed Portal territories.


  \subsection{A player's turn}

    \hspace{-2em} During his turn, a player can perform several actions,
    which can be done in any order but not simultaneously:
    \vspace{-1.3em}
    \begin{itemize}
        \item Discarding any card from his hand (any number of times)
        \item Buying cards (any number of times)
        \item Managing his fully controlled territories (only once per turn)
        \item Launching attacks on territories (only once per turn)
        \item Making a breakthrough (only once per turn,
        if a player-controlled territory was conquerred)
    \end{itemize}
    
    \vspace{-0.7em}
    At the end of a player's turn, \textbf{any player} whose hand contain less than 5
    cards \textbf{can replenish his hand to 5 cards}.
    
    \hspace{-2em} \textbf{IMPORTANT:
    When replenishing one's hand, one may always choose to replenish one's hand to
    less than 5 cards.
    }


\newpage
  \subsection{Drawing and discarding cards}
    
    During his turn, a player may choose discarding card as an action.
    In this case, he may discard any number of cards from his hand.

    When a player must draw more cards from his deck
    (whether to replenish his hand or because of a card's effect)
    or reveal more cards from the top of his deck than his deck contains cards,
    that player must replenish his deck before drawing or revealing the cards.
    To do so, he shuffles his discard pile and puts the cards under his deck.


  \subsection{Resources}
  
    Territories may hold any number of resources (there is no upper limit),
    which can be of 7 types:
    
    \vspace{-1.3em}
    \begin{itemize}
        \item Shards (white tokens)
        \item Plants (green tokens)
        \item Animals (red tokens)
        \item Ore (grey tokens)
        \item Positronium (yellow tokens)
        \item Omninutrients (brown tokens)
        \item Inhabitants (pink tokens)
    \end{itemize}
    
    \vspace{-0.7em}
    Resources on a territory are materialized by resource tokens.
    Counters may be put under resource tokens to count the number of
    resources of the corresponding type.
    
    Fully controlling a territory allows the player to use the resources it holds,
    as well as to produce or destroy those resources during territory management.
  
  
  \subsection{Buying cards}

    During his turn, a player may choose to buy cards.
    He may buy Energy Crystal cards (in-universe: refining shards),
    buy Tactical cards (in-universe: making use of the terrain)
    or buy Unit cards (in-universe: asking for reinforcement).
    
    Buying cards first implies paying a cost, by using resources and Energy Crystals
    (or only resources in the case of common cards).
    Once the cost has been paid, the card is put on top of the player's discard pile.
    
    \hspace{-2em} \textbf{IMPORTANT:
    Each civilization relies heavily on one type of resource, or
    \textit{valuable resource} (see the color of the cubes on their unit cards).
    As such, the only resources a player can produce or use are Shards and
    his civilization's valuable resource.
    }


\newpage
  \subsection{Managing one's fully controlled territories}
    \subsubsection{Management}
    \vspace{-1em}
    Once per turn, a player can manage the territories he fully controls.
    For each territory, he can choose to:
    \vspace{-1.3em}
    \begin{itemize}
        \item Produce resources (see below),
        \item Destroy resources: he removes all resource tokens on the territory,
        \item Do nothing.
    \end{itemize}
  
    \subsubsection{Resource production}
    \vspace{-1em}
      \hspace{-2em} Resource production works differently depending on the
      territories involved:
      \vspace{-1.3em}
      \begin{description}
          \item[Portal territory] \hfill \\
          No resource can be produced on Portal territories,
          \item[Heart of Eden] \hfill \\
          The player removes all tokens from his Portal territory,
          converts each of them into a resource token of his choice,
          and places them on the Heart of Eden tile,
          \item[Normal and face-down territories] \hfill \\
          The player places 1 valuable resource token on each of his territories
          which produces resources and whose color is the same as that resource.
          He places 1 shard token on each of his territories which produces resources and
          have an increased-shard-production token on them.
          Then the player counts his normal or face-down territories which produce,
          takes as many resource token of any combination of shard and valuable resource,
          and places them on the territories he controls.
      \end{description}

    \subsubsection{Production capacities}
      \vspace{-1em}
      Some effects involve the \textbf{\textit{production capacity}} of
      a territory in a type of resource.
      In the case of a \textbf{normal territory} and \textbf{face-down territories},
      the production capacity is simply the amount of that type of resource that
      the territory is able to produce,
      \textit{i.e.} 2 if it is a normal territory in the resource of its color,
      and 1 otherwise.
      
      In the case of \textbf{Portal territories and the Heart of Eden},
      their production capacities are considered to be 2 in their controllers'
      valuable resource types, and 1 in any other type of resource.
    
      \textbf{Territories outside of the map} are considered to have a
      production capacity of 1 in every type of resource.
      That is to say, if an effect is played on a tile on the border
      of the map and mentions the production capacity of the neighbouring tiles,
      the "missing" tiles are also considered to have a production capacity of
      1 in every type of resource.


\newpage
\subsection{Launching an attack}

    Once per turn, a player may choose frontiers between territories he fully controls
    and territories he does not control and start a battle on each of those frontiers.
    Each of those battles takes place on a separate "front"
    (see next section for more information on combat).

    Once combat is over, if the attacking player won battles on some fronts,
    he takes control of the corresponding territories by removing the Occupation tokens of
    the previous controller of the conquered territories (if any) and putting his own
    Occupation tokens on it.
    At that time, he may move resources as he wishes between the two territories on each
    side of the victorious fronts.
    
    \hspace{-2em} \textbf{
    IMPORTANT: winning on one front is enough to take control of a territory.
    }


\subsection{Making a breakthrough}

    If at least one territory controlled by other players was conquered during his attack
    phase, the attacking player earns the right to perform a breakthrough.

    \hspace{-2em} To make a breakthrough:
    \vspace{-1.3em}
    \begin{enumerate}
        \item the attacking player may replenish his hand, \label{break1}
        \item the attacking player selects any number (possibly 0) of frontiers between
        those territories taken from other players and territories he does not control,
        \item the attacking player launches an attack on the selected frontiers,
        \item defending players (if any) may discard any number of cards from their hands,
        then replenish them,
        \item the battles are resolved,
        \item if this is the first time this step is reached and player-controlled
        territories were conquered in the previous step, the attacking player may choose
        to go back to step \ref{break1} (\textit{double-breakthrough}),
        \item the attacking player discards all cards in his hand.
    \end{enumerate}


\subsection{Conquering a Portal territory}

    If a player conquers a Portal territory, the tile is placed on the "destroyed" side.
    The player who lost his Portal territory is eliminated from the game
    and all of his territories become partially controlled.
    
    The player who conquered the Portal territory takes the counter tokens
    on it and places them on his own Portal territory.
    He also ears the right to perform an additional breakthrough from that territory.


\newpage
\section{Combat}

  \subsection{Attacking a player's fully controlled territory}
  
    \hspace{-2em} In the case where there is a single front,
    against a player-controlled territory:
    \vspace{-1.3em}
    \begin{enumerate}
        \item The attacking player plays a unit card from his hand on the front,
        hereby launching the attack,
        \item If he has not stopped defending, \label{attack2}
        the defending player may either play a unit on this front or stop defending,
        \item If he has not stopped attacking, \label{attack3}
        the attacking player may either play a unit on this front or stop attacking,
        \item Repeat step \ref{attack2} and \ref{attack3} until both player have
        stopped attacking or defending,
        \item All players have stopped attacking or defending,
        and can now trigger their units delayed abilities, \label{attack5}
        \item On each side, sum the strengths of all units, \label{attack6}
        The attacking player wins the battle on this front if the total strength
        of his units is strictly more than that of the defending player's,
        \item Post-combat phase (see below).
        \item Unit-destruction phase (see below)
    \end{enumerate}
    

  \subsection{Attacking a wild territory of a partially controlled territory}
    
    When a wild territory or a partially controlled territory is attacked,
    \textbf{there is no defending player on that front}.
    The battle on that front happens exactly as in the previous section,
    but as if there was a defending player who stopped defending in step 2.
    
    When step \ref{attack6} is reached, the battle is won by the attacking player
    if the total strength of his units is at least equal to the distance
    of the wild territory to the closest Portal territory.
    

  \subsection{Attacking multiple territories}
    
    The battle on each front is similar to what is described above,
    but each battle take place in parallel:
    each step has to be performed on each front before going to the next step.
    Players can choose the order in which they perform a step on the various fronts.
    When a player stops attacking or defending on a front, only this front is affected.


\newpage
\subsection{Effects and abilities}

    Cards played during combat don't simply add strength to their side.
    They also have effects and abilities that greatly impact how the battle unfolds.
    A distinction must be done between permanent effects and triggered abilities.

    \begin{description}
        \item[Permanent effects] \hfill \\
        	Those are additional rules that apply to a card,
            like special buying conditions, ways to use it from your hand, 
            increased force in battle under certain conditions for a unit, etc.
        \item[Triggered abilities] \hfill \\
            Units in battle allow their owner to trigger their abilities at
            specific times, which are events that can have varying effects.
            A player can always choose whether to trigger an ability or not.
            
            Triggered abilities may be of 4 types,
            depending on when they can be triggered:
            \begin{description}
        	        \item[Immediate abilities]
                    	may be triggered just after the unit is added to the battle.
        	        \item[Delayed abilities]
                    may be triggered in step \ref{attack5},
                     when all players have stopped attacking or defending,
	                before the actual end of the battle in step \ref{attack6}.
        	        \item[Reaction abilities] may be event-based or condition-based.
	                \begin{description}
                        \item[Event-based:]
                        they may be triggered each time a specific event occurs
                        (example of event: "collapse of this unit").
                        \item[Condition-based:]
                        they may be triggered each time the event
                        "the card is played while the condition is satisfied" or the
                        event "the condition just became satisfied" occurs.
                        (example of condition: "no card in your hand").
                    \end{description}
        	        \item[Post-combat abilities] may be triggered at the beggining of the
                    post-combat phase.
            \end{description}
    \end{description}
    
    If several abilities can be triggered at the same time,
    they don't preempt each other and may be triggered successively instead.
    The order in which they are triggered is decided by the player that triggers them.
    If several players want to trigger an ability at the same time,
    the attacking player may trigger his first,
    then the priority goes to the one that plays after him, and so on.
    
    \hspace{-2em} \textbf{
    IMPORTANT: An ability of the form "\textbf{If} A\textbf{:} do B" can only be
    triggered if the condition A is satisfied.
    }
    
    \hspace{-2em} \textbf{
    IMPORTANT: An ability of the form "Do A \textbf{to} do B" can only be triggered
    if you are able to perform A.
    }
    

\newpage
  \subsection{Post-combat phase and unit-destruction phase}
    
    Some units'abilities allow them to \textbf{\textit{get rid of}} a unit.
    Units targetted by those abilities \textbf{\textit{collapse}},
    which means they are removed from the battle and put aside into a pile
    of \textbf{\textit{collapsed units}} until the unit-destruction phase.
    
    \textbf{During the post-combat phase,} various abilities may be triggered.
    Afterwards, all units that have not collapsed are discarded.
    On each front, the units are discarded in the order they have been played.
    
    \textbf{During the unit-destruction phase,} collapsed eternal units
    (i.e. cost-6 units) are first discarded.
    Then, each player may choose to \textbf{\textit{rescue}}
    any number of his collapsed units.
    To rescue a unit, the player uses as many valuable resources as this unit's cost.
    Rescued units are discarded while all other collapsed units are destroyed.
    

  \subsection{Technical terms}
  
    \begin{description}
        \item[Allied territory] \hfill \\
        A front represents the battle on a frontier between two territories.
        The \textbf{\textit{allied territory}} is the territory on
        your side of the frontier.
        \item[Active front] \hfill \\
        An \textbf{\textit{active front}} is a front on which there is still a player
        attacking or defending.
        \item[Moving a unit] \hfill \\
        \textbf{\textit{Moving a unit}} is the action of removing a unit from the front
        where it has been played and adding it to another front without triggering
        its immediate ability.
        Units can only be moved to active fronts.
        \item[Dooming and saving units] \hfill \\
        Some unit's abilities allow them to \textbf{\textit{doom}} a unit.
        At that moment, the \textbf{\textit{doomed units}}'s card is turned
        sideways to the right.
        All doomed units collapse during the post-combat phase,
        after all post-combat abilities have been played.
        Some other unit's abilities allow them to \textbf{\textit{save}} a unit,
        which means canceling the effects of its dooming and making the unit
        undoomed again.
        At that moment, the unit's card is turned back to the left.\\
        Please note that \textbf{undoomed units cannot be saved},
        and that \textbf{doomed units cannot be doomed again}.
        \item[Producing/Using temporary resources on a front] \hfill \\
        \textbf{\textit{Mana}} is a temporary resource that may be produced
        and spent during battle.\\
        \textbf{Producing a temporary resource} is materialized by placing counters
        besides your units of a front (your \textit{\textbf{reserve}} on this front).\\
        \textbf{Using a temporary resources} means removing tokens from your reserve on
        this front.
        At the end of the post-combat phase, all reserves are emptied.
        \item[Harvesting resources] \hfill \\
        \textbf{\textit{Harvesting a territory's resource}} is the action of
        putting on the allied territory as many new tokens of that type of resource
        as the harvested territory's production capacity in that resource.
    \end{description}


\section{Frequently Asked Questions}

    \hspace{-2em}
    \textbf{Q:} Do I have to discard my hand at the end of my turn?
    \newline
    \textbf{A:} No, you don't have to. But you can use your "discard" action to discard
    any card you don't want to have in your hand before replenishing it.

    \hspace{-2em}
    \textbf{Q:} I just conquered another player's territory, which had resources on it.
    Can I use those resources ?
    \newline
    \textbf{A:} You can use your valuable resource and shards, but not other resources.

    \hspace{-2em}
    \textbf{Q:} Can I attack wild territories during a breakthrough?
    \newline
    \textbf{A:} Yes, you can. A breakthrough allows you to attack any territory in contact
    with the player-controlled territories you conquered, whether those attacked
    territories belong to the same player or to another player, and even wild territories.

    \hspace{-2em}
    \textbf{Q:} Can I use cards from my hands between an attack and a breakthrough?
    \newline
    \textbf{A:} Yes, you can. A breakthrough is a distinct action from an attack and
    need not be played in succession.

    \hspace{-2em}
    \textbf{Q:} Can I discard cards from my hands when I transform my breakthrough into
    a double breakthrough ?
    \newline
    \textbf{A:} No, you can't.

    \hspace{-2em}
    \textbf{Q:} My "Soldier" unit's immediate ability added another "Soldier"
    unit to the front. Can I trigger that unit's immediate ability?
    \newline
    \textbf{A:} Yes, you can.

    \hspace{-2em}
    \textbf{Q:} My "Rocket Roll" unit's immediate ability says I can destroy a card
    in my discard pile to purchase a unit and add it to my hand.
    Do I have to pay for this unit?
    \newline
    \textbf{A:} Yes. You need to both discard Energy Crystal cards from your hand
    and remove resource tokens from your territories to pay that unit card's cost.
    
    \hspace{-2em}
    \textbf{Q:} My "Charge Leader" unit's reaction ability added another
    "Charge Leader" unit to the front. Can I trigger that unit's reaction ability?
    \newline
    \textbf{A:} Yes, you can.
    
    \hspace{-2em}
    \textbf{Q:} My "Shaman" unit has several immediate abilities. Can I choose the order
    in which I trigger them ?
    \newline
    \textbf{A:} Yes, you can.
    
    \hspace{-2em}
    \textbf{Q:} My "Shaman" unit's immediate ability says it affects my unit "for the
    rest of the battle". What happens if the "Shaman" unit collapses?
    \newline
    \textbf{A:} The "Shaman" unit's effect on your units stays effective until
    the end of the battle, whether the Shaman remains in battle or not.
    
    \hspace{-2em}
    \textbf{Q:} My "Warlord" unit's post-combat ability says that I can save it.
    Doesn't that mean that it can't be destroyed?
    \newline
    \textbf{A:} No, it doesn't. This ability means that you can remove its doomed status
    if it was doomed during battle.
    It does not mean that you can automatically rescue it after it collapses.
    Please also note that post-combat abilities are triggered abilities and can only
    be triggered when the unit is still in battle.
    
    
    



\end{document}


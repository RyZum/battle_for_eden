\documentclass[a4paper]{article}

\usepackage[a4paper, margin=1in]{geometry}
\usepackage[english]{babel}
\usepackage[utf8x]{inputenc}
\usepackage{indentfirst}

\author{Adrien Matricon}

\setlength{\parskip}{1em}

\begin{document}

\title{How to Play Battle For Eden}
\maketitle


    Civilizations rise and fall across the multiverse.

    Meanwhile, a place is left untouched, never knowing magic or technology.
    Its peaceful fairy inhabitants live in harmony with nature,
    never knowing hunger or lacking anything, for it is a land of abundance.

    A land filled with all kinds of riches, a land filled with all kinds of resources,
    a land in which the ever so rare energy crystals are commonly found
    and in whose heart energy flows in its purest form: a land called Eden!

    When energy builds up so much in Eden that natural portals 
    connecting it to the multiverse suddenly open,
    it is time for civilizations to go to war!
    War for crystals! War for resources! This is the battle for Eden!



\section{Choose a color and a civilization}

    Each player picks a color for his Portal tile and frontier tokens,
    then chooses a civilization.
    
    The color choice is purely cosmetic,
    but the civilization choice matters a lot: each civilization plays different,
    which leads to an asymmetric gameplay and endless replayability!
    
    There are 7 types of resources in Battle For Eden,
    each associated to a color:
    crystal shards (white), plants (green), animals (red), ore (grey),
    positronium (yellow), omninutrients (brown) and inhabitants (pink).
    Each civilization is built on one type of resources,
    which will be referred to as its \textit{civilization resource},
    and which will be the resource type that civilization will need
    to improve its army.
    
\section{Choose the first player}
    Choose the first player in anyway you like,
    and give that player the first player token.


\newpage
\section{Choose a scenario}

   The first contact between civilizations can happen in various ways.
   Three scenarios are playable, each with its own winning conditions
   as well as its own set of slight deviations from the standard rules
   (which are described in the following sections).
   
    \begin{description}
        \item[Supremacy] \hfill \\
            All civilizations want Eden for themselves.
            They all try to destroy the portals connecting the other civilizations to Eden
            so that they can be the only ruler.
            
            \textbf{Deviation from the standard rules:}
            None.
            
            \textbf{Winning condition:}
            The game ends when all portals but one have been destroyed.
            The player whose portal is the last one standing wins.


        \item[Unbalance] \hfill \\
            While most civilizations want peaceful coexistence on Eden,
            one of them is more powerful than the others
            and prefers to fight for supremacy.
            
            \textbf{Deviation from the standard rules:}
            The first player replenishes his hand to 6 cards instead of 5.
            He is the only player that can look at the tiles he drew before choosing
            where to place them on the map.
            He also starts in a different situation (see section TODO).
            
            \textbf{Winning condition:}
            The game ends if either the first player's portal has been destroyed
            (in which case everyone else wins),
            or if his portal is the last one standing (in which case he wins).
            
            \textbf{Variant of the scenario:}
            The other civilizations were not that peaceful after all!
            You may ignore the restriction on looking at the tiles before placing them.
            When the first player's portal is destroyed,
            the game does not end and the winning condition switches to those
            of the Supremacy scenario.
            


        \item[King of Eden] \hfill \\
            All civilizations want peaceful coexistence on Eden,
            but their arrival awoke a powerful dragon: the King of Eden !
            Now they have to slay the dragon if they want to survive.
            
            \textbf{Deviation from the standard rules:}
            Players don't look at the tiles they drew before choosing
            where to place them on the map.
            The King of Eden is on the map and play turns (see section TODO).
            
            \textbf{Winning condition:}
            The game ends when the King of Eden is slain (in which case everyone wins)
            or when all portals have been destroyed (in which case everyone loses).
            
            \textbf{Variant of the scenario:}
            The civilizations were not that peaceful after all!
            You may ignore the restriction on looking at the tiles before placing them.
            When the King of Eden is slain,
            the game does not end and the scenario switches to Supremacy.
    \end{description}



\newpage
\section{Create the Map}

    After crossing the portals, all civilizations discover Eden.
    What the world looks like and where the portals open is determined by the players.
    
    \subsection{Understand tiles, territories and resources}
        The map is divided into territories, which are materialized by hexagonal tiles.
        Those territories can be of three types:
         \vspace{-1.3em}
        \begin{itemize}
            \item \textbf{Ordinary territories}, materialized by mono-color tiles,
                make most of the map.
                Their color determine a type of resources in which they are rich
                (with black meaning none).
                Controlling an ordinary territory brings a player 1 resource at every
                production phase (see section \ref{prod}), or 2 if the territory is rich
                in that player's civilization resource.
                Ordinary territories tile may have a token on them symbolizing richness
                in crystal shards (see section \ref{tile-distribution} and \ref{map-tiles}),
                in which case they bring an additional resource each turn to their controller.
            
            \item \textbf{The Heart of Eden} is a mysterious mountain located near
                the center of the map (see \ref{map-tiles}).
                Controlling the Heart of Eden allows a player to extract
                pure energy at every production phase (see section \ref{prod}),
                which can be spent to purchase extremely powerful cards.
            
            \item \textbf{Portal territories} are what connects the various worlds
                to Eden.
                They are therefore the players' starting points on the map,
                and the players whose portals have been destroyed are eliminated from
                the game.
        \end{itemize}   

    \subsection{Distribute the tiles}
        \label{tile-distribution}
        Each player receives 1 ordinary territory tile of each color (including black),
        places them face down in front of him, and shuffles them.
        Each player then receives the portal territory tile of his color
        as well as 1 ordinary territory tile rich in his civilization resource,
        and places them face up in front of him.
        
        \textbf{Note:} If a player's civilization resource is crystal shards,
        that player can choose the color of the face-up ordinary territory tile
        he receives.
        Whenever he places a tile of this color on the map,
        place a crystal shard richness token on it.
      
    \subsection{Place tiles on the map}
        \label{map-tiles}
        Place the Heart of Eden tile at the center of the table as the first
        territory on the map.
        Starting with the first player, players take turn adding tiles to the map.
        
        At his turn, a player draws one ordinary territory tile
        (either the face-up one or one of the face-down ones),
        reveals it and places it face up in contact with another tile on the map.
        If this causes an ordinary territory tile to be in contact with at least
        4 different types of non-black ordinary territory tiles,
        place a crystal shard richness token on that territory.
        
        After placing the ordinary territory tile,
        the player can choose to remove his portal territory tile from the map
        (if it was already there),
        then to place it anywhere in contact with another tile on the map.
        
\newpage
        Keep in mind that:
        \vspace{-1.3em}
        \begin{itemize}
            \item At any time, a portal territory tile cannot be at a distance
                (number of frontiers one has to cross to go from one to the other)
                of 3 or less of the Heart of Eden,
                or at a distance of 4 or less of another portal territory tile,
            \item At any time, a portal territory cannot be in contact of 3 or more
                ordinary territory tile,
            \item A portal territory tile cannot be removed
                if it disconnects one part of the map from the other,
            \item A player cannot place an ordinary territory tile at a distance
                or 3 or less of his portal territory tile (if it has been placed on the map).
        \end{itemize}
        
        After placing his last ordinary territory tile,
        if a player's portal is not the map, he must place it.
        If he cannot do this without breaking the above rules,
        he is given a black ordinary territory tile,
        which he has to place in a way that allows him to place his portal tile on the map.
        If it is still not possible, he is given an additional black ordinary tile, etc.
        
        Once every player has placed his last ordinary territory tile,
        one last round allows players to remove then place their portal territory tiles
        if they want to.
        
        The first player to have placed his portal territory tile in its final position
        takes the first player token and gives it to the player of his choice
        (which can be himself).
    

\section{Settle near the portals}
    \subsection{Territories}
        Every player takes control of their respective portal territory
        and of the territories in direct contact with it.
        Controlled territories are materialized using the frontier tokens
        of the player's color.
        
    \subsection{Deck}
        Every player places all unit cards of their civilization in front of them,
        face up (generally as one or more piles), to constitute their \textit{World}.

        Similarly, \textit{common cards} (Small Energy Crystal, Medium Energy Crystal,
        Huge Energy Crystal, Setback and Opportunity) and \textit{technology cards}
        (Energy Shield, Temporary Rift, Improved Portal, Unmatched Power)
        are placed face up somewhere where every player can access them.
        
        Each player then takes  3 cost-1 units from his World as well as
        1 Medium Energy Crystal card and from 3 to 5 Small Energy Crystal cards
        (at their convenience) from the common cards,
        shuffles those cards and places them face down
        as a pile called their \textit{deck}.
        
        Every player finally creates their \textit{hand} by drawing cards from the top
        of their deck until they have up to 5 cards in their hand.
        Drawing cards in this way is called \textit{replenishing} one's hand.
        Whenever a player should draw or reveal more cards from his deck
        than it contains, he shuffles his \textit{discard pile}
        (which is an originally empty face up pile besides one's deck)
        and puts the cards at the bottom of his deck before doing so.
        
    
\newpage
    \subsection{Resources, purchases}
        \label{how-to-purchase}
        Every player receives 2 resources (which can each be either a crystal shard
        or their civilization resource) and places them as he wishes on his territories.
        
        Players can choose to \textit{purchase} cards,
        whether they are common cards or cards from their World.
        Purchasing cards is a way to get new cards into your discard pile,
        and therefore into your future deck.
        
        The opposite of  purchasing a card is called \textit{destroying a card}.
        It means putting a card back in the pile where it comes from:
        destroyed common cards go into the corresponding piles,
        while destroyed unit cards go back to their World.

        Purchasing a card is done in two steps:
        \vspace{-1.3em}
        \begin{itemize}
            \item first, the player pays the cost of the card by Energy Crystals, 
                 pure energy, and/or resources of the right type
                (note: the multi-color cube symbol corresponds to resources
                which can be of any type, but must all be of the same type):
                \vspace{-0.5em}
                \begin{itemize}
                    \item \textit{using N pure energy} means to decrease the
                    \textit{pure energy count} on the Heart of Eden by N.
                    This pure energy count starts at 0,
                    and increases when players extract pure energy from the Heart of Eden
                    (see section \ref{prod}).
                    They keep track of their pure energy count using counter
                    tokens placed on their portal territory,
                    \item \textit{using N Energy Crystals} means to discard Energy Crystal
                        cards whose total value is at least N.
                        During the purchase phase (see section \ref{purchase}),
                        they are discarded from cards which have been previously set aside.
                        In all other contexts that allow purchases (like now),
                        they are discarded from the player's hand,
                    \item \textit{using N resources} means to remove N resources
                        from one's territories.
                \end{itemize}
            \item then the player puts the purchased card in his discard pile.
        \end{itemize}
        
        Note: players don't get to replenish their hands after purchases.


\section{To battle!}

    Now that all civilizations have crossed the portals and come to Eden,
    it is time for them to harvest their territories and to go to war to expand
    their reach.
    
    From here on, the game enfolds as a cycle of 3 different phases.
    
    \subsection{Production phase: the pulse}
        \label{prod}
        The Heart of Eden pulses energy throughout Eden,
        making the land rich in resources.
        
        Ordinary territories produce resources:
        every player gets a combination of their choice of crystal shards
        and their civilization resource:
        \vspace{-1.3em}
        \begin{itemize}
            \item 1 resource for each ordinary territory they control,
            \item 1 additional resource for each of those territories
                rich in their civilization resource,
            \item 1 additional resource for each ordinary territory they control
                with a crystal shard richness token on them.
        \end{itemize}
        
        The players may then place those resources on their territory tiles
        (possibly including the Heart of Eden tile or their Portal tile)
        in any way they want.
        
        The player who controls the Heart of Eden extracts pure energy from it:
        he increases the \textit{pure energy count} by 1
        (using counter tokens).
        
        Note: Production normally happens simultaneously for all players.
        Should conflict arise as to who does what first,
        consider that everything happens as if the players went through
        the production phase one after the other, starting with the first player
        (\textit{i.e.} the one who has the first player token).

    \subsection{Expansion phase}
        \label{expansion}    
        Starting with the first player,
        players take turn expanding their territories.
        
        A player's turn is a succession of attack subphases.
        During each attack subphase, the player may start battles
        on up to 2 frontiers of his territories (possibly 0),
        either against \textit{wild territories}
        ---~which are territories that no player control~---
        or against territories controlled by any given player.
        Those battle enfold as described in section \ref{battle-rules}.
        
        A player can originally perform a single attack subphase.
        Taking one or more territories from another player during an attack subphase
        earns him the right to perform an additional one.
        In other words, as long as the player keeps attacking territories controlled by
        other players and winning, he can keep attacking.
        
        At the beginning and end of each attack subphase,
        the player can choose to set aside any number of cards from his hand.
        Setting aside cards helps players managing their hand
        and is needed to purchase new units from their world
        (see section \ref{how-to-purchase}).
        
        The player may also choose to discard any number of cards from his hand
        before and after each battle (if there is any), and at the end of his first
        attack subphase (even if there is no battle).
        
        At the very end of each subphase:
        \vspace{-1.3em}
        \begin{enumerate}
            \item the player takes the cards he set aside back into his hand if he earned
                     another attack subphase,
            \item in all cases, both the player and his opponent (if any)
                     replenish their hands.
        \end{enumerate}
        
        \textbf{Note:} During each additional attack subphase,
                              the player cannot set aside more cards than he got back from
                              those set aside during the previous subphase.
        
    
    \subsection{Purchasing phase}
        \label{purchase}
        Behind the front lines, each civilization prepares for the battles to come.
        They refine crystal shards into energy crystals that they can use to
        sustain more troops, accommodate reinforcements,
        and think of ways to get an edge over their enemies.
        
        Every player can use their pure energy, their resources
        and the Energy Crystal cards that they had set aside to make purchases
        (see section \ref{how-to-purchase} for the details on how to purchase cards).
        
        After the purchases, the players draw all remaining set-aside cards into their
        hands, then discard as many.        
        This phase ends with the first player giving the first player token
        to the next player.
        
        Note: All players normally purchase cards simultaneously.
        Should conflict arise as to who does what first,
        consider that everything happens as if the players went through
        the purchasing phase one after the other, starting with the first player.


\newpage
\section{Battle rules}
    \label{battle-rules}
    
    \subsection{General Principle}
        Battle may happen on up to two several frontiers at the same time.
        The territory on a player's side of the frontier is called his \textit{allied territory},
        while the territory on the other side is called his \textit{enemy territory}.
        The battle taking place at any given frontier is referred to as a \textit{front}.
        Units played by the player on his side of the front are called his
        \textit{allied units}, while the units played by the other player on the other
        side of the front are called his \textit{enemy units}.
        
        The attacking player adds unit cards from his hand to the fronts
        (see section \ref{battle}).
        If the total strength of his units on one front is more than that territory's
        defense, he takes control of the corresponding enemy territory
        (note: this means that when attacking one territory on two fronts,
        victory on only one of them is enough to take control of that territory).
        Resources that were on the conquered territory stay on that territory.
        
    \subsection{Defense of a wild territory}
        The defense of a wild territory depends on its distance
        to the closest portal territory:
        it is equal to 1 when the distance is 2,
        then increases by 2 each time the distance increases by 1
        (which means that territories at a distance of 2, 3, 4, 5
        to the closest portal territory have a base defense of  1, 3, 5, 7
        respectively).
        
    \subsection{Defense of a player-controlled territory}
        \label{base-defense}
        All territories controlled by a player have a \textit{base defense},
        which is 5 at the beginning of the game and increases by 1
        (until a maximum of 14)
        each time the player successfully defends one of his territories.
        Players may place counter tokens on their portal territory tiles to keep
        track of the base defense of their territories.
        
        Each territory controlled by a player can be either \textit{fully-controlled}
        or \textit{partially-controlled}.
        A player's fully-controlled territories are made up of its portal territory
        (if his portal has not been destroyed), and recursively of all of his territories
        which are in contact with his fully-controlled territories.
        The rest of his territories are partially-controlled.
        
        The defense of a partially-controlled territory is equal to its base defense.
        When a fully-controlled territory is attacked,
        the player may either choose to rely on its base defense
        or use unit cards from his hand to defend it manually (see section \ref{battle}).
        
    \subsection{Special cases}
        \begin{itemize}
            \item When a player loses his portal territory,
                     his portal is destroyed (the tile is flipped),
                     and he is eliminated from the game.
                     His territories are still considered under his partial control
                     (\textit{i.e. } their defense is determined by his base defense
                     and bonuses).
                     Resources that were on the portal territory tile stay on it.
            \item When a player loses control of the Heart of Eden,
                     he is allowed to use the pure energy on it to purchase technology cards.
        \end{itemize}
        
\newpage
    \subsection{Bonuses}
        \begin{itemize}
            \item If there are at least 5 resources on a territory which are either
                crystal shards or its controller's civilization resource,
                that territory gets +1 strength in both attack and defense.
            \item If a territory is in contact with at least 1 other territory
                in the above situation (controlled by the same player),
                it also gets +1 strength in both attack and defense.\\
                (note: this means that a territory that satisfies both conditions
                can get a total bonus of +2 strength)
        \end{itemize}
        
    \subsection{Battle}
        \label{battle}
        \begin{enumerate}
            \item The attacking player chooses a frontier on which he wishes to start
                a battle, then plays a unit from his hand to create a front.
                He may do this a second time if he wants to create a second front.
            \item If there is a defending player,
                and if he was already attacked during this expansion phase,
                he may discard any number of cards from his hand then replenish it.
            \item If there is a defending player,
                he plays a unit from his hand on each front he chose to defend himself.
            \item On each front on which he has not finished attacking,
                the attacking player may either stop attacking or add a unit
                from his hand under the bottom row of his allied units on this front
                (during battle, on each side of each front,
                units are placed one under the other in rows; 
                some abilities (see section \ref{abilities}) allow units to be placed
                on the same row or take advantage of the units placement).
                \label{battle4}
            \item On each front on which he has not finished defending,
                the defending player may either stop defending or add a unit
                from his hand under the bottom row of his allied units on this front.
            \item Go back to step \ref{battle4} until both players have stopped
                attacking and defending.
                \label{battle6}
            \item Players count their total strength (including possible bonuses).
                On each front where the attacking player's strength is more than
                the defending player's (or than the base defense of the territory
                if the defending player did not defend it himself),
                the attacking player takes control of his enemy territory.
                \label{battle7}
            \item \textbf{Post-combat phase:}\\
                Each player may choose to return some of his units to his World.
                To do so, he must use as many civilization resources
                as the sum of those units' costs.
                The rest of his units are discarded,
                from top row to bottom row and from left to right within a row.
            \item \textbf{Unit-destruction phase:}\\
                Some unit's abilities (see section \ref{abilities}) allow them
                to \textit{get rid of} participating units during the battle.
                Units targeted by those abilities \textit{collapse}:
                they are removed from the battle and put aside
                into a pile of collapsed units until the unit-destruction phase.\\
                During this phase, each player's cost-6 collapsed units are discarded,
                while the rest of their collapsed units are destroyed.
                They may \textit{rescue} any number of them by using
                as many civilization resources as the sum of their costs,
                in which case those units are discarded instead.      
        \end{enumerate}

\newpage
    \subsection{Effects and abilities}
        \label{abilities}
        Cards played during combat don't simply add strength to their side.
        They also have effects and abilities that greatly impact how the battle unfolds.

    \begin{description}
        \item[Permanent effects] are additional rules that apply to a card,
            like special buying conditions, ways to use it from your hand, 
            increased strength in battle under certain conditions for a unit, etc.
        \item[Abilities] are actions that the card allows its owner
            to perform at specific times, which can have varying effects.
            A player can always choose whether to trigger an ability or not.
            They may be of 4 types:
            
            \begin{description}
        	        \item[Immediate abilities]
                       may be triggered just after the unit is added to the battle.
        	        \item[Delayed abilities]
                       may be triggered between steps \ref{battle6} and \ref{battle7}
                       of the battle, that is to say when all players have stopped
                       attacking and defending but before counting their total strength.
        	        \item[Reaction abilities] may be event-based or condition-based.
	                    \begin{description}
                            \item[Event-based reaction abilities]
                                may be triggered each time a specific event occurs
                                (example of event: "this unit collapses").
                            \item[Condition-based reaction abilities]
                                may be triggered each time the events
                                "the card is adde to the battle while the condition is satisfied" or
                                "the condition just became satisfied" occur.
                                (example of condition: "your hand is empty").
                        \end{description}
        	        \item[Post-combat abilities]
        	            may be triggered at the beginning of the post-combat phase,
        	            before discarding units or returning them to their World.
            \end{description}
    \end{description}
    
    Note: if several abilities can be triggered at the same time,
    they don't preempt each other and may be triggered successively instead.
    The order in which they are triggered is decided by the player that triggers them.
    If several players want to trigger an ability at the same time,
    the priority goes to the attacking player.
    
    
\newpage
\section{Frequently Asked Questions}

    \hspace{-2em}
    \textbf{Q:} I put some cards aside during my first turn (of the expansion phase),
    and now I don't have enough cards in my deck to replenish my hand to 5 cards.
    Is that normal?
    \newline
    \textbf{A:} Yes, it is.

    \hspace{-2em}
    \textbf{Q:} I replenished my hand to less than 5 cards at the end of my turn,
    but after the purchasing phase I now have cards in my discard pile.
    Can I replenish my hand to 5 cards?
    \newline
    \textbf{A:} No, you can't.

    \hspace{-2em}
    \textbf{Q:} Can I choose not to attack any territory during my turn?
    \newline
    \textbf{A:} Yes, you can.

    \hspace{-2em}
    \textbf{Q:} The opponent just won the battle and took my territory.
    Can I use the resources on it during the post-combat phase
    or unit-destruction phase, before the territory becomes his?
    \newline
    \textbf{A:} No, you can't. The territory shifts controllers before just
    before the post-combat phase.

    \hspace{-2em}
    \textbf{Q:} Some abilities have things like "if X: Y" or "if X: do Z" in them,
     what does it mean?
    \newline
    \textbf{A:} It means that the condition X has to be satisfied for effect Y to apply,
    or to perform action Z.

    \hspace{-2em}
    \textbf{Q:} Some abilities have things like "do X to do Y" in them,
    what happens if I can't do X?
    \newline
    \textbf{A:} Doing X is some form of cost you need to pay in order to be able to
    do Y. If you don't do X, you're not allowed to do Y either.

    \hspace{-2em}
    \textbf{Q:} Some abilities have things like "Level X: A. Level Y: B in them,
     what does it mean?
    \newline
    \textbf{A:} This is equivalent to "[if this unit's level is X or more: A]
     OR [if this unit's level is Y or more: B]".

    \hspace{-2em}
    \textbf{Q:} My "Infantry Squad" unit's immediate ability makes me reveal
    the top 2 cards of my deck, but after discarding/playing the first card,
    I can't do anything with the second one. Is that normal?
    \newline
    \textbf{A:} Yes, it is. It is meant to give you some information on what's
    on top of your deck. If you don't discard/play the first card,
    this also allows you to change the order of those cards when you place them back
    on top of your deck.

    \hspace{-2em}
    \textbf{Q:} My "Infantry Squad" unit's immediate ability added
    a "Captain" unit to the front. Can I trigger that unit's immediate ability?
    \newline
    \textbf{A:} Yes, you can. That unit was just added to the battle,
    which is when immediate abilities can be triggered.
    In similar cases, other kinds of abilities can be triggered
    (for example when a "Charge Leader" unit is added to the front when your
    hand is empty, or when a "Skeleton" unit is added to the front
    between steps \ref{battle6} and \ref{battle7} of the battle).

    \hspace{-2em}
    \textbf{Q:} My "Digging Machine" unit's immediate ability says I can harvest
    neighbouring territories' ore. Does it mean that I get fewer resources when
    I play it from a territory on the edge of the map?
    \newline
    \textbf{A:} No, it doesn't. When looking at neighbouring territories,
    everything happens as if there were black tiles beyond the edges of the map.

    \hspace{-2em}
    \textbf{Q:} My "Rocket Roll" unit's immediate ability says that I can destroy a card
    in my discard pile to get the right to immediately purchase a unit
    and add it to my hand. Do I have to pay for this unit?
    \newline
    \textbf{A:} Yes, you do.
    You need to both discard Energy Crystal cards from your hand
    and remove resource tokens from your territories to pay that unit card's cost.
    
    \hspace{-2em}
    \textbf{Q:} My "Shaman" unit has several immediate abilities.
    Can I choose the order in which I trigger them ?
    \newline
    \textbf{A:} Yes, you can. Your units' abilities are triggered in the order in which
    you want to trigger them, and only if you want to trigger them.
    
    \hspace{-2em}
    \textbf{Q:} My "Shaman" unit's immediate ability says it affects my units
    "for the rest of the battle". What happens if the "Shaman" unit collapses?
    \newline
    \textbf{A:} The effect applies until the end of the battle,
    whether the Shaman is still on the front or not.
    
    
\newpage
\section{Glossary}
  
    \begin{description}
        \item[Active Front] \hfill \\
            A \textit{front} is said to be \textbf{\textit{active}} if there is still a player
            who has not stopped attacking or defending.
            
        \item[Allied Territory] \hfill \\
            On a given \textit{front}, a player's \textbf{\textit{allied territory}}
            is the territory on his side of the frontier.
            
        \item[Allied Units] \hfill \\
            A player's \textbf{\textit{allied units}} are the units he played on his side
            of the \textit{fronts}.
            
        \item[Base Defense] \hfill \\
            See section \ref{base-defense}.
            
        \item[Civilization Resource] \hfill \\
            A player's \textbf{\textit{civilization resource}} is the type of resource
            on which his civilization is based, and which that player needs to purchase
            units from his \textit{World}.
            
        \item[Collapse] \hfill \\
            When a unit \textbf{\textit{collapses}},
            it is removed from the battle and added to its player's collapsed unit pile,
            where it stays until the unit-destruction phase.
            
        \item[Destroy] \hfill \\
            \textbf{\textit{Destroying a card}} is the action of putting a card back
            where it comes from: in the player's World for unit cards and
            in the corresponding pile for common cards.
            
        \item[Doom] \hfill \\
            A \textbf{\textit{doomed unit}} is a unit which still takes part in the
            battle but \textit{collapses} during the post-combat
            phase, just before discarding the participating units.
            Dooming a unit is materialized by turning that unit's card
            sideways to the right.
            Doomed units cannot be doomed again.
            
        \item[Enemy Territory] \hfill \\
            On a given \textit{front}, a player's \textbf{\textit{enemy territory}}
            is the territory on the other side of the frontier.
            
        \item[Enemy Units] \hfill \\
            A player's \textbf{\textit{enemy units}} are the units played on the other side
            of the \textit{fronts} by opposing players.
            
        \item[Front] \hfill \\
            A \textbf{\textit{front}} is a battle taking place at a given frontier.
            Both attacking and defending players play units on their side of the front.
            
        \item[Full Control] \hfill \\
            A player is said to have \textbf{\textit{full control}} over his portal territory
            if his portal has not be destroyed.
            Any territory under that player's control which is in contact with
            a territory that the player fully controls is under his full control as well.
            
        \item[Get Rid Of] \hfill \\
            \textbf{\textit{Getting rid of a unit}} means making that unit
            \textit{collapse}.
            
        \item[Harvest] \hfill \\
            \textbf{\textit{Harvesting a resource from a territory}} is the action of
            placing as many resources of that type on the \textit{allied territory}
            as the harvested territory's \textit{production capability} in that resource.
            
        \item[Level] \hfill \\
            \textbf{\textit{Levels}} are a way to represent the fact
            that a unit has been enabled to do more than it could originally
            (by becoming stronger, having advantageous circumstances, etc).
            All units are at level 1 when they arrive into battle,
            then this level may be modified by some abilities or effect.
            Other abilities or effects may depend on the unit's level.
            
        \item[Mana] \hfill \\
            \textbf{\textit{Mana}} is a type of \textit{temporary resource}.
            
        \item[Move] \hfill \\
            \textbf{\textit{Moving a unit}} means to remove that unit from its
            \textit{row} and place it somewhere else.
            By default, moving a unit to a \textit{front} places it
            under the bottom \textit{row} of that \textit{front}
            (thus creating a new row).
            If the moved unit keeps the \textit{status} it had before being moved.
            Moving a unit does not count as adding a new unit to the battle.
            
        \item[Neighbouring Territories] \hfill \\
            A territory's \textbf{\textit{neighbouring territories}} are the 6 territories
            with which it shares a frontier.
            If some of the territory's frontiers are on the edge of the map,
            everything happens as if there were black tiles
            on the other side of those frontiers.
            
        \item[Partial Control] \hfill \\
            A \textbf{\textit{partially controlled territory}} is territory controlled
            by a player, but not \textit{fully controlled} by that player.
            
        \item[Production capability] \hfill \\
            The \textbf{\textit{production capability}} of a territory in one type
            of resource measures the richness of this territory in that resource.
            Ordinary territories have a production capability of 2 in the type of resource
            in which they are rich (\textit{i.e.} the resource of their tile color,
            and none for black tiles), and 1 in all other types of resources.
            The Heart of Eden has a production capability of 2 in every type
            of resource.
            Portal territories have a production capability of 1 in every type of resource,
            but it becomes 2 in their owner's \textit{civilization resource}
            if the portals on them are still standing.
            
        \item[Purchase] \hfill \\
            See sections \ref{how-to-purchase} and \ref{purchase}.
            
        \item[Pure Energy Count] \hfill \\
            See section \ref{how-to-purchase} and \ref{prod}.
            
        \item[Reconfigure] \hfill \\
            \textbf{\textit{Reconfiguring a unit}} is the action of replacing a unit
            on its row by a combination of units (possibly only one unit)
            of the same cost from its player's World.
            Those units have the same \textit{status} as the reconfigured unit,
            and their immediate abilities cannot be triggered.
            
        \item[Replenish] \hfill \\
            \textbf{\textit{Replenishing one's hand}} is the action of drawing
            cards from the top of one's deck until having up to 5 cards in one's hand.
            
        \item[Rescue] \hfill \\
            \textbf{\textit{Rescuing a unit}} is the action of using resources
            during the unit-destruction phase to discard a unit in the collasped unit
            pile instead of \textit{destroying} it.
            
        \item[Resistant] \hfill \\
            A \textbf{\textit{resistant unit}} is a unit which becomes \textit{doomed}
            instead of \textit{collapsing} if it is undoomed.
            
        \item[Restore] \hfill \\
            \textbf{\textit{Restoring a unit}} is the action of removing
            its \textit{doomed} status, which is materialized by turning
            that unit's card back to the left.
            Undoomed units cannot be restored.
            
        \item[Row] \hfill \\
            A player's units are added to a \textit{front} one under the other,
            and are therefore placed in \textbf{\textit{rows}}.
            Some abilities allow units to be placed on the same row or
            take advantage of the units placement.
            If a row becomes empty, it disappears and the units in the rows
            above and below are now in successive rows.
            
        \item[Status] \hfill \\
            The \textbf{\textit{status}} of a unit is everything that affects how
            that unit behaves, except for its placement.
            It includes things like "having triggered an ability" and "being doomed".
            
        \item[Temporary Resources] \hfill \\
            \textbf{\textit{Temporary resources}} are resources which do not appear
            on the map, but can be produced and used during battle.
            Temporary resources produced on a given \textit{front} can only be
            used on this \textit{front}.
            Counter tokens are used to materialize how many are available on each front,
            and those remaining at the end of the battle are lost.
            
        \item[Using Energy Crystals/Pure Energy/Resources] \hfill \\
            See section \ref{how-to-purchase}.
            
        \item[Wild Territory] \hfill \\
            A \textbf{\textit{wild territory}} is a territory
            which is not under any player's control.
    \end{description}
    


\end{document}

